\documentclass[12pt,a4j]{jarticle}
\title{\LaTeX の基本形}
\author{a00919 山形 太郎}
\date{2006 年 3 月 31 日}

\begin{document}
\section{実習4.2}
\subsection{3.5打ち込んだ通りに出力する}
\begin{verbatim}
    #include <stdio.h>
    int main()
    {
        int i=2;
            printf("%d %d\n", i, i*i);
        return 0;
    }
    \end{verbatim}
    ここで、\verb|#include| はプリプロセッサへの命令で... また、\verb+%d+で整数型の変数を表す.    
\subsection{3.6箇条書き}
上杉家の名君として次の二人が有名。
\begin{description}
    \item[上杉謙信] 1530 -- 1578 没。米沢上杉家の藩祖。名将と謳われる。
    \item[上杉鷹山] 1751 -- 1822 没。第9代米沢藩主。中興の名君。
\end{description}

The latest CD by ''PUR'' consists of 4 songs with the title:
\begin{enumerate}
    \item Ouverrt \"ure
    \item Hold me tight
    \item If you hear this Tango
    \item Adventure land.
\end{enumerate} 

山形大学米沢キャンパス組織図:
\begin{itemize}
    \item 工学部(6学科)
    \begin{itemize}
        \item 情報・エレクトロニクス学科
        \item システム創成工学科
        \item (他4学科略)
    \end{itemize}
    \item 大学院理工学研究科(6専攻)
    \begin{itemize}
        \item 数理情報システム専攻
        \item 情報・エレクトロニクス専攻
        \item (他4専攻略)
    \end{itemize}
\end{itemize}

\subsection{3.7中央寄せとスペーシング}
強調のため、ある単語または文を中心に置くには
\begin{center}
    center 環境
\end{center}
を用いる。

\subsection{3.8表組み}
\begin{center}
    \begin{tabular}{r|lc}
        \hline
        国名 & 首都 & 通貨 \\
        \hline
        日本 & 東京 & 円 \\
        イギリス & ロンドン & ポンド\\
        アルゼンチン & ブエノスアイレス & ペソ\\
        \hline
    \end{tabular}
\end{center}

\begin{table}[h]
    \caption{各国の首都と通貨}
    \begin{center}
        \begin{tabular}{r|lc}
        \hline
        国名 & 首都 & 通貨 \\
        \hline
        日本 & 東京 & 円 \\
        イギリス & ロンドン & ポンド\\
        アルゼンチン & ブエノスアイレス & ペソ\\
        \hline
        \end{tabular}
    \end{center}
\end{table}

\end{document}