\documentclass[12pt,a4j]{jarticle}

\usepackage[dvipdfmx]{graphicx}

\begin{document}
$ z = x +yi $を複素数とする。領域$D$は次の不等式:
\[
    D = \{z = x + yi||z| \le 1,|z-1| \ge 1,|z+1| \ge 1,y > 0\}
\]
で定義されているとする。この時$D$の頂点を求めたい。
$|z|= \sqrt{x^{2}+y^{2}}$なので、$D$の境界は
\begin{eqnarray}
    x^{2}+y^{2} &=&1 \label{eq:xyintA}\\
    (x-1)^{2}+y^{2}&=&1\label{eq:xyintB}\\
    (x+1)^{2}+y^{2}&=&1\label{eq:xyintC}
\end{eqnarray}
で与えられる。(\ref{eq:xyintA})から(\ref{eq:xyintB})を引くことにより、
\[
    2x+1=0
\]
を得る。すなわち$x=\frac{1}{2}$となる。これを(\ref{eq:xyintA})に代入すると
\[
    (\frac{1}{2})^{2}+y^{2}=\frac{1}{4}+y^{2}=1
\]
すなわち$y^{2}=\frac{3}{4}$となる。$y>0$なので、$y=\frac{\sqrt{3}}{2}$を得る。同様に、(\ref{eq:xyintA})から(\ref{eq:xyintC})を引くことにより、
\[
    2x+1=0
\]
を得る。すなわち$x=-\frac{1}{2}$となる。これを(\ref{eq:xyintA})に代入して$\left(x,y\right)=\left(-\frac{1}{2},\frac{\sqrt{2}}{2}\right)$を得る。同様に、(\ref{eq:xyintB})から(\ref{eq:xyintC})を引くことにより、
\[
    4x=0
\]
を得る。これを(\ref{eq:xyintB})に代入することにより、$(x,y)=(0,0)$となる。
 以上により、図1で与えられる$D$の各頂点は$x=0,\frac{1}{2}+\frac{\sqrt{3}}{2}i,-\frac{1}{2}+\frac{\sqrt{3}}{2}i$の三点で与えられる。
\begin{figure}[h]
    \centering
    \includegraphics[width=6cm]{./pic.png}
    \caption{Dの領域}
\end{figure}
\end{document}