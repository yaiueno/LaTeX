\documentclass[10.5pt,a4j]{jarticle}
\usepackage[dvipdfmx]{graphicx}
\usepackage{amssymb, amsmath}
\usepackage{cite}
\usepackage{url}

\title{{\Huge 情報科学演習レポート}}
\author{\empty}
\date{\empty}

\begin{document}
\maketitle

\begin{center}
  \vspace{115mm}
  \begin{description}
  	\item \Large 実験課題名 \quad \underline{*************}
	\item \Large 実験実施日 \quad \underline{2024年**月**日,**月**日}
	\item \Large レポート提出日 \quad \underline{2024年**月**日}
%	\item \Large レポート再提出日 \quad \underline{2024年**月**日}
	\item \Large 学生番号 \quad \underline{2251****}
	\item \Large 報告書提出者 \quad \underline{******}
  \end{description}
\end{center}

\thispagestyle{empty}
\clearpage
\addtocounter{page}{-1}

\newpage
\section{目的}
本演習の目的は****。

\section{実験装置・環境}
本演習で使用した環境は以下の通りである。

コンピュータ環境:*****

ソフトウェア:*****

\section{方法}
\subsection{実習 4.1}
hello.texに以下のコード:\\
\verb|--------------------|

\begin{verbatim*}
\documentclass[12pt,a4j]{jarticle}
\begin{document}
\subsection{演習3.5}
次は C のサンプルである。
\begin{verbatim}
#include <stdio.h>
int main()
{
int i=2;
printf("%d %d\n", i, i*i) ;

return 0;
}
\end{verbatim}
ここで、\verb|#include| はプリプロ
セッサへの命令で... また、\verb+%d+
で整数型の変数を表す.

\subsection{演習3.7}
強調のため、ある単語または文を
中心に置くには
\begin{center}
center 環境
\end{center}
を用いる。
\end{document}
\end{verbatim*}
\verb|--------------------|\\
を入力する.その後,hello.texをコンパイルし,その結果を画面上に表示させる.

\subsection{実習 4.2}

\subsection{実習 4.3(a)}

\subsection{実習 4.3(b)}

\subsection{実習 8.1}

\subsection{実習 8.2}

\subsection{実習 8.3}

\subsection{実習 8.4}

\section{演習結果}
\subsection{実習4.1}
「方法」で説明した実習4.1のソースコードをコンパイルすると,画面上に以下のように表示された.\\
\verb|--------------------|
\subsubsection{演習3.5}
次は C のサンプルである。
\begin{verbatim}
#include <stdio.h>
int main()
{
int i=2;
printf("%d %d\n", i, i*i) ;

return 0;
}
\end{verbatim}
ここで、\verb|#include| はプリプロ
セッサへの命令で... また、\verb+%d+
で整数型の変数を表す.

\subsubsection{演習3.7}
強調のため、ある単語または文を
中心に置くには
\begin{center}
center 環境
\end{center}
を用いる。\\
\verb|--------------------|

\subsection{実習 4.2}

\subsection{実習 4.3(a)}

\subsection{実習 4.3(b)}

\subsection{実習 8.1}

\subsection{実習 8.2}

\subsection{実習 8.3}

\subsection{実習 8.4}

\section{考察}
本演習では,*****。

\begin{thebibliography}{9}
\bibitem{ref1} 情報・エレクトロニクス学科(情報・知能コース)編,「\LaTeX{}による文書作成入門」,\url{https://ecsylms1.kj.yamagata-u.ac.jp/webclass/course.php/2452513/manage/?acs_=7fdb2584}, 2024.
\bibitem{ref2} 奥村晴彦, 黒木裕介著,「[改訂第8版]LaTeX2ε美文書作成入門」,技術評論社,2020.
\end{thebibliography}

\end{document}  
